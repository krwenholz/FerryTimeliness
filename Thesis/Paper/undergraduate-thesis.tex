% !TEX TS-program = pdflatex
% !TEX encoding = UTF-8 Unicode

% This is a simple template for a LaTeX document using the "article" class.
% See "book", "report", "letter" for other types of document.

\documentclass[11pt]{article} % use larger type; default would be 10pt

\usepackage[utf8]{inputenc} % set input encoding (not needed with XeLaTeX)

%%% Examples of Article customizations
% These packages are optional, depending whether you want the features they provide.
% See the LaTeX Companion or other references for full information.

%%% PAGE DIMENSIONS
\usepackage{geometry} % to change the page dimensions
\geometry{a4paper} % or letterpaper (US) or a5paper or....
% \geometry{margins=2in} % for example, change the margins to 2 inches all round
% \geometry{landscape} % set up the page for landscape
%   read geometry.pdf for detailed page layout information

%%% Line Spacing
\usepackage{setspace}
\onehalfspacing

\usepackage{graphicx} % support the \includegraphics command and options

% \usepackage[parfill]{parskip} % Activate to begin paragraphs with an empty line rather than an indent

%%% PACKAGES
\usepackage{booktabs} % for much better looking tables
\usepackage{array} % for better arrays (eg matrices) in maths
\usepackage{verbatim} % adds environment for commenting out blocks of text & for better verbatim
\usepackage{subfig} % make it possible to include more than one captioned figure/table in a single float
\usepackage{amsmath, amssymb, amsthm, lastpage}
% These packages are all incorporated in the memoir class to one degree or another...

%%% HEADERS & FOOTERS
\usepackage{fancyhdr} % This should be set AFTER setting up the page geometry
\pagestyle{fancy} % options: empty , plain , fancy
\renewcommand{\headrulewidth}{0pt} % customise the layout...
\lhead{Team \# 16677}\chead{}\rhead{Page \thepage\ of \pageref{LastPage}}
\lfoot{}\rfoot{}\cfoot{}

%%% SECTION TITLE APPEARANCE
%\usepackage{sectsty}
%\allsectionsfont{\sffamily\mdseries\upshape} % (See the fntguide.pdf for font help)
% (This matches ConTeXt defaults)

%%% ToC (table of contents) APPEARANCE
%\usepackage[nottoc,notlof,notlot]{tocbibind} % Put the bibliography in the ToC
%\usepackage[titles,subfigure]{tocloft} % Alter the style of the Table of Contents
%\renewcommand{\cftsecfont}{\rmfamily\mdseries\upshape}
%\renewcommand{\cftsecpagefont}{\rmfamily\mdseries\upshape} % No bold!

%%% END Article customizations

%%% The "real" document content comes below...
%\date{}

\begin{document}
\begin{titlepage}
    \vspace*{\fill}
    \begin{center}
      \Huge{Where's my Ferry?}\\[0.5cm]
      \Large{Support Vector Machines for Modeling Ferry Tardiness}\\[0.4cm]
      \today
    \end{center}
    \vspace*{\fill}
  \end{titlepage}
\newpage
\vspace*{\fill}
\tableofcontents
\vspace*{\fill}
\newpage

\section{Preliminaries}
\label{sec:prelims}
%Possibly include some comparisons of our model to existing research if any.

% Example figure
%\textbf{Figure \ref{fig:ex}}.
%\begin{figure}[h]
%  \centering
%  \includegraphics[scale=.6]{}
%  \caption{}
%  \label{fig:ex}
%\end{figure}
\subsection{Project goals}
\label{sec:goals}

% Project goals
%  Attack a messy problem
%  Learn some _practical_ AI
%  Work with big data
%  <img src="images/correlation.png">
%  image height="70%" width="70%" src="images/ferry.jpg">

\subsection{Project components}
\label{sec:components}

% All the parts of a good time 
% Hyperplanes
% Ferries
% Big (messy) data
% Tardiness
% 
% This is an introduction slide.  I just need to quickly go over
%  who I worked with, that I wanted to learn about AI in practical 
%  situations and do something real with real data, what my project 
%  was (Senior Project/Thesis), what's the problem, and what my 
%  solution was.

\section{Classification as a model}
\label{sec:classification}

% Classification as a model
% Tardy or not
% Features are intuitive
%  Route
%  Vessel
%  Weather
% Hyperplanes are (mostly) intuitive
%

\subsection{SVMs for tardiness}
\label{sec:svms_tardiness}
  
% Discuss the major benefits of SVM for my project: features, 
%  the drawing analogy, use for big data. 
% 
% Common uses of classification
%  Spam
%  Cancer
%  Facial recognition
%  Airplane traffic
%  <img style="float:right;margin:4px;"width="50%" height="50%" 
%  src="images/male-female.png">
%  
%  (email, cancer pictures, use in a paper regarding 
%  airlines) 
% 
%  A 2008 paper discussed the use of SVM for predicting a sort of 
%  "bad day" for air traffic control (weather based mostly).  
%  78\% accurate on predicting need to use some sort of software 
%  and 83\% for predicting delays in air traffic.  They used a 
%  super computer.
% 
%  ## Ho! An example!
%  <img src="images/basic_svm_data.png">
%  2. What an SVM is
%      1. How an SVM works
%      2. How it relates to the ferry problem
%          1. Separating late as >3 minutes past estimate
%          2. The data as points in >33 space
%      3. The complication of fitting such a curve
% 
% 

\subsection{Linear classifiers}
\label{sec:lcs}
%  ## Linear classifiers
%  * Feature vector
%  * Weight vector (line, plane, hyperplane)
%  * Sign of dot product is the "class"
% /script>
% /section>
% 

\subsubsection{Feature vectors}
\label{sec:feature_vecs}
%  ## Feature Vectors
%       Time (seconds),
%       Humidity (percent),
%       Temperature (F),
%       Wind Speed (mph),
%       Wind Direction (degrees off N),
%       Vessel Name,
%       Route Name
% 
%       360
%       10,
%       3,
%       6,
%       17,
%       USS Enterprise,
%       Pt. D/Tahlequah
% 
%  Two of these are not like the others.
% 
%  ## Categorical to numeric
%      USS Enterprise,
%      S.S. Minnow,
%      Millenium Falcon,
%      Death Star
%  
%  An average trip for Han Solo:
% 
%      0,
%      0,
%      1,
%      0
% 
% ## WOAH! 
%  ## How do we get the weights?
% 

\subsubsection{Training}
\label{sec:training}
%  ## Learning with </br>Support Vector Machines 
%  * Training data
%    - Mix of both classes
%    - "Big"
%  * A machine to train the weight vector
%    - Draws those lines many times
%    - Knows when to stop
%  * Mighty hardware to do the calculations
%    - Exponentially difficult
%    - Long running process
% 

%  ## How will we know it works?
%  ### Test data for validation
%  - Disjoint from training data
%  - Mix of both classes
%  - Also "big"
% 

%  ## Visualizing an SVM
%  <img src="images/Svm_separating_hyperplanes.png">
%  2. What an SVM is
%      1. How an SVM works
%      2. How it relates to the ferry problem
%          1. Separating late as >3 minutes past estimate
%          2. The data as points in >33 space
%      3. The complication of fitting such a curve
% 
% 
%  ## A real world problem:
%  <img width="50%" height="50%" src="images/tough-separation.png">
% 

\subsubsection{Kernels}
\label{sec:kernels}
%  ## Kernels: bending space
%  <img src="images/kernel-machine.png">
% 
%  ## Tuning the kernel
%  <img src="images/grid_search.png">
%  Like a brute force search
%  Try many (almost all) possibilities
%  Need to bound it
%  Use model accuracy as a performance metric
%  Measured through cross validation
% 

\subsection{LibSVM}
\label{sec:libsvm}
%  ## LibSVM
%  Chih-Chung Chang and Chih-Jen Lin
% 
%  "Working set selection using second 
%  </br>order information for training SVM"
% 
%  <image height="40%" width="40%" src="images/chih-jen.gif">
%  3. Use of LibSVM
%      1. Why not write my own?
%          1. Time
%          2. Correctness
%          3. Performance
%      2. Use in other papers/popularity
%      3. Ease of use of the library (reasonably documented)
%
\section{The problem with ferries}
\label{sec:problem}
%    ## Why do we care about ferries?
%    ### <p class="fragment roll-in">Popular</p>
%    ### <p class="fragment roll-in">Complex</p>
%    ### <p class="fragment roll-in">Lots of data</p>
%    Talk about what the ferry system is (WSDOT), 
%    and how many people use it, how the problem 
%    extends.
% 

\subsection{Washington State's ferries}
\label{sec:wsdot}
%  A portrait of the system</h2>
%       Established in 1951</li>
%        22 million passengers per year</li>
%        20 terminals, 10 routes, 23 vessels</li>
%        Largest ferry system in the world for vehicles</li>
%        From Victoria to Pt. Defiance (~130 miles)</li>
%        159,811 sailings in 2012 (12,764 in December)</li>
%        Almost 10,000,000 vehicles in 2011</li>
%     <img width="110%" 
%      src="images/route-map-overview.gif" align="right">
%  Go over the details of the ferry system.
%  1. history
%  1. Boats
%  2. # Of trips per day
%  3. Routes
%  4. Places serviced
%  6. Design and planning
%  7. Resources for travelers
%    1. Vessel Watch
%    2. Pamphlets
%    3. General knowledge
%    4. Email alerts
%  * Mention that ~95% of boats are on time.  This poses some interesting
%    questions about what we can actually learn.
% 
%  96.4% on time 
%  Best: Pt. Defiance/Tahlequah and Edmonds/Kingston (99.5%)
%  Worst: Anacortes/San Juans (88.9%)
% 

\subsection{Where the data comes from}
\label{sec:data_origins}
%  ## Sources and shades of data
%  * <p class="fragment roll-in">Scraping web pages</p>
%  * <p class="fragment roll-in">WSDOT VesselWatch</p>
%  * <p class="fragment roll-in">NOAA Tacoma Narrows weather station</p>
%  <img src="images/vesselwatch.png">
%  1. Accessibility of the Vessel Watch data
%      1. Briefly discuss the web page scraper
%          1. Using a simulated web browser
%          2. Accounting for "botched" grabs
%      2. Using a request for the data
%          1. Discuss what I actually received
%          2. Discuss the quality and amount of data
%  2. Grabbing weather data
%      1. Grabbed from a deeply hidden NOAA archive
%      2. Manually downloaded files
%      3. Why the Tacoma Narrows station
%          1. Completeness of dates
%          2. Completeness of records
% 
%  ## Ferry data (340,902)
%      Kittitas, Mukilteo, Clinton, 9/1/2010 0:00, 9/1/2010 0:01,
%      Mukilteo - Clinton, 9/1/2010 0:13, 9/1/2010 0:14, 9/1/2010
% 
%  </br>
%  ## Weather data (1184 &times; 24)
%      94274,20110101,0053,12,CLR, ,10.00, , , ,25, ,-3.9, ,22, ,-5.4, ,
%      16, ,-8.9, , 69, , 0, ,000, , , ,29.69, ,1, ,002, ,30.04, ,AA, , ,
%      ,30.03, 
% 
%  ## Weather variables 
% 
% talk about the many weather variables I got from NOAA.
% 
%  ## Data monsters 
%    * Python for everything
%    * 1,034 weather recordings missing wind data
%    * 13,220 vessel trips removed from lack of weather data (~ 4%)
%    * Stages of formatting
%        - Combine all weather data files from NOAA
%        - Join ferry and weather data
%        - Handle categorical variables
%        - Make it all SVMable
%  1. Discussion of data quality
%    1. Missing values
%    2. Ill formatted values
%    3. Generally consistent (huge amount so a few mistakes is okay)
%  2. Separated into 290,000 for training and 30,000 for testing
%  2. Suite of Python scripts to cut out bad data
%  3. Scripts to join weather and ferry data
%  4. Performance issues encountered
% 

\section{Using SVM as a model}
\label{sec:results}
 
%  ## Goal refresh 
%  * Find order in the mess: Which features matter?
%  * Do something practical: Can we use the model?
%  * Work with big data: Was all of the data necessary/useful?
% 90 accuracy isn't expected!  This would be weird in most cases.
% This is because it can't be perfect, there is a maximum iter 
% threshold, it's optimization, noise is still noise, 
% 
% Mention the 95% accurate fact and how we can basically be correct
% Most of the time by guessing on time.  We want a model to be similarly
% accurate.
% 

\subsection{The first pass}
\label{sec:firstpass}
%  ## SVM is easy right?
%  # <p class="fragment roll-in">No. 60%</p>
%  WHY DID I DO BETTER???
%  1. Accuracy comes from test data (disjoint from train data)
%  1. Needing to fit c and gamma parameters to get better curves
%  2. Duration of runs and sheer size of data sets
% 

\subsection{General results}
\label{sec:results}
%  ## Results: Arrivals
%  *  Full: 87.0%
%  *  No Weather: 88.0%
%  *  No Departure: 60.0%
%  *  No Departure, No Weather: 59.4%
%  Full = Departure estimate and actual,
%  arrival estimate, vessel name, route, all weather
%  1. How do we know it's correct?
%      1. Classification versus regression: I want a "useful" machine
%      2. Test sets vs. training sets
%  2. Review the numbers of the results
%      1. Results before param fitting of c and gamma
%      1. Departures
%          1. Weather
%          2. No weather
%      2. The more interesting: arrivals
%          1. Weather
%          2. No weather
% 
%  ## Results: Departures
%  * Full: 76.3%
%  * No Weather: 75.7%
%  Full = Departure estimate, vessel name, route, all weather
% 
%       <img width="70%" src="images/route-map-overview.gif" align="right">

\subsection{The problem with weather}
\label{sec:weather_prob}
% ## How about that weather?
%  Weather/No Weather
% Without information about the departure, the arrival time
%    predictions were strictly less than 68% accuracy.
%  1. Possible issues
%      1. Location of station
%      2. Poor quality of data
%      3. My error
%      4. Ferries really are consistent enough to not need weather
%  2. Running individual routes
%      1. Closest to Tacoma Narrows
%      2. Middleish (randomly selected)
%      3. Geographically furthest

% TODO: Add in the actual weather data to discuss

\section{Closing thoughts}
\label{sec:summary}
%  ## In summary
%  * This is a "cute" model.
%  * Hyperplanes are more than an awesome word.
%  * The ferry system is full of data we can mine.
%  * SVM is like a black box.
% 

\subsection{Remaining questions}
\label{sec:questions}
%  1. Making an actual application
%      1. State of Java web frameworks
%      2. Need to scrape or grab user data
%      3. Update SVM
%  2. Are the schedules online adjusted as time goes by
%  3. Mining the data for patterns
% 

\subsection{Extensions}
\label{sec:extensions}
%  ## Future(ish) work
%  * Multi-class SVM for "tardiness categories"
%  * Regression analysis
%  * Graphical application


\newpage 

\bibliographystyle{amsplain}
\bibliography{undergraduate-thesis.bib}

\end{document}

